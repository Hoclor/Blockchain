\documentclass[11pt,a4paper]{article}
\usepackage[margin=0.5in, bottom=1.75cm]{geometry}
\usepackage{caption}
\usepackage{tabu}
\usepackage[hidelinks]{hyperref}

\captionsetup[figure]{labelfont=bf}
\captionsetup[table]{labelfont=bf}

\bibliographystyle{ieeetr}

\title{\vspace{-2em}Blockchain and Cryptocurrencies Coursework}
\author{pbqk24}
%\date{}

\begin{document}
	\maketitle
	
	\vspace{-3em}
	
	\section{Mining Puzzles}
	
	\subsection{Proof of Work}
	The requested information for Proof of Work is as follows:
		
	\begin{table}[h!]
		\centering
		\begin{tabu} to 1.0\linewidth {|r|p{8cm}|X[l]|}
			\hline
			Index&Information&Value\\
			\hline
			1&User ID&pbqk24\\
			\hline
			2&Block hash target calculated (hex)&000003e7fc1800000000000000000000\newline00000000000000000000000000000000\\
			\hline
			3&Nonce value (int)&$2171906$\\
			\hline
			4&Number of (double) hashes performed&$2171907$ (one per nonce tried, starting at 0)\\
			\hline
			5&Estimate for mining time for difficulty $= 1$&$58629$ seconds\\
			 &Estimate for mining time for difficulty $= 7454968648263$&$4.37*10^{17}$ seconds\\
			\hline
		\end{tabu}
		\caption{Mining Puzzles: Information Requested}
		\label{table_mining_puzzles_pow}
	\end{table}

	\noindent The equations used for calculating the mining time estimates $t$ are as follows:
	
	\begin{equation}
	target = \frac{target_{initial}}{difficulty}
	\end{equation}
	
	\begin{equation}
	h_q = \frac{h_{space}}{target}
	\end{equation}
	
	\begin{equation}
	t = h_q * t_h
	\end{equation}
	
	Where $h_q$ is the estimated number of hashes required to be performed to find a valid hash, $target$ is the target value, $t_h$ is the estimated time per hash calculated by timing how long it takes to perform a large number of hashes (e.g. $10^9$) and finding the average. $t_h$ was computed as $1.365*10^{-5}$ and used for both calculations.
	
	For difficulty of $1$:
	
	$$target = \frac{00000000FFFF0000000000000000000000000000000000000000000000000000}{1}$$
	
	$$h_q = \frac{2^{256}}{00000000FFFF0000000000000000000000000000000000000000000000000000} = 4295032833$$
	
	$$t = 4295032833 * 1.365*10^{-5} = 58629$$
	
	For difficulty of $7454968648263$:
	
	$$target = \frac{00000000FFFF0000000000000000000000000000000000000000000000000000}{7454968648263} = 3.62*10^{54}$$
	
	$$h_q = \frac{2^{256}}{3.62*10^{54}} = 3.2 * 10^{22}$$
	
	$$t = 3.2 * 10^{22} * 1.365*10^{-5} = 4.37*10^{17}$$
	
	\subsection{Proof of Stake}
	
	The requested information for Proof of Stake is as follows:
	
	\begin{table}[h!]
		\centering
		\begin{tabu} to 1.0\linewidth {|r|p{7cm}|X[l]|}
			\hline
			Index&Information&Value\\
			\hline
			6&ECDSA public key (hex)&4f045a6cfacb3e67e7c5d4ddfb9f1acfe7d6ddda\newline c29869734cce5218cdab24e2d2cc72601138d6f\newline 324464df7691f819cd14e8b3752d9c463e5162a\newline ad37393ca0\\
			\hline
			7&Signature of "Hello world" (hex)&eae12ab8fdbeb5635ac45edbfceb999907a5b090\newline 42eeddbd9a07a744f656b3ac7e00124086256e5\newline caf86539e68186742d593e5e8b537b9f6d7ee055\newline 57c2ef68a\\
			\hline
			8&Signature used in calculating hit (hex)&aa2974089248c51977f63350c3aad2757b935d68\newline a236dd777621c3ed879657d2c5e1f01d22ad16b\newline b2d37f1c2567d1daeccd4e3f1a45201f53291e2e\newline ba9e9bea3\\
			\hline
			9&Hit value (hex)&a135a0781dd3c0f6\\
			\hline
			10&Time (s) that you could forge a new block&394\\
			\hline
		\end{tabu}
		\caption{Mining Puzzles Information}
		\label{table_mining_puzzles_pos}
	\end{table}
	
	\noindent The equations used for calculating the hit value are as follows:
	
	\begin{equation}
	hit_{full} = hash(sign(S_g))
	\end{equation}
	
	Where the hit value is the first 8 bytes of $hit_{full}$, and $S_g$ is the generation signature for the previous block:
	
	$$S_g = 9737957703d4eb54efdff91e15343266123c5f15aaf033292c9903015af817f1$$
	
	$$sign(S_g) = aa2974089248c51977f63350c3aad2757b935d68a236dd777621c3ed879657d2c5e1f01d22ad16bb2d37f1c2567d1daeccd4e3f1a45201f53291e2eba9e9bea3$$
	
	$$hash(sign(S_g))[:8] = a135a0781dd3c0f6$$
	
	\noindent The time in seconds when you would be able to forge a new block was calculated using the following equations:
	
	\begin{equation}
	T = T_b * B_e
	\end{equation}
	
	\begin{equation}
	t_{forge} = \frac{hit}{T}
	\end{equation}
	
	Where $T$ is the target value (independent of time passed), $T_b$ is the base target value, $B_e$ is your effective balance, and $t_{forge}$ is the time when you can forge a new block (rounded up, in seconds):
	
	$$T = 1229782938247303 * 24 = 29514790517935272$$
	
	$$t_{forge} = \frac{11616367251628998902}{29514790517935272} = 394$$
	
	\section{Interacting with bitcoin-testnet}
	
	The requested information for Task 2 is included in Table \ref{table_bitcoin_testnet} and \ref{table_bitcoin_testnet_transactions} below.
	
	\begin{table}[h!]
		\centering
		\begin{tabu} to 1.0\linewidth {|r|l|X[l]|}
			\hline
			Index&Information&Value\\
			\hline
			1&User ID&pbqk24\\
			\hline
			2&blockchain.com links&This is included in Table \ref{table_bitcoin_testnet_transactions} below\\
			\hline
			3&Transaction explanations&This is included in Table \ref{table_bitcoin_testnet_transactions} below\\
			\hline
			4&bitcoin-testnet address&mvcM9NV5hesnSUNpZGZ9Pyt3PK8xDFmCTp\\
			\hline
			5&100 Satoshi Tx ID&bfb4090faca08fe0f1ca46883cef1bf7e7860a4df6637acb7962741db0206f29\\
			&chain.so hyperlink&\href{https://chain.so/tx/BTCTEST/bfb4090faca08fe0f1ca46883cef1bf7e7860a4df6637acb7962741db0206f29}{https://chain.so/tx/BTCTEST/bfb4090faca08fe0f1ca46883cef1bf7e78\newline 60a4df6637acb7962741db0206f29}\\
			\hline
			6&Proof of Burn Tx ID&611186bc95e29fc70127fce35bd9c8a7f86c46f00aecd9df4765731d27f64d31\\
			&chain.so hyperlink&\href{https://chain.so/tx/BTCTEST/611186bc95e29fc70127fce35bd9c8a7f86c46f00aecd9df4765731d27f64d31}{https://chain.so/tx/BTCTEST/611186bc95e29fc70127fce35bd9c8a7f8\newline6c46f00aecd9df4765731d27f64d31}\\
			\hline
			7&Hex used as script&6a067062716b323\\
			&Explanation of hex script&This script is equivalent to `OP\_RETURN 06 7062716b323'. This causes the transaction to always be marked invalid, as OP\_RETURN always outputs \emph{fail}. The `06' signals that the next 6 bytes are to be pushed onto the stack, and `7062716b323' is the result of converting `pbqk24', my USER ID, from ASCII to hex.\\
			\hline
		\end{tabu}
		\caption{Bitcoin-Testnet Information}
		\label{table_bitcoin_testnet}
	\end{table}

	\begin{table}[h!]
		\centering
		\begin{tabu} to 1.0\linewidth {|r|l|X[l]|}
			\hline
			Tx&Aspect&Explanation\\
			\hline
			1&blockchain.com link&\href{https://www.blockchain.com/btc/tx/6c260da65fe98b08b27b80f8481c6f0ae34252921f6c34d327172472c5b419d7}{https://www.blockchain.com/btc/tx/6c260da65fe98b08b27b80f8481c6f0ae\newline 34252921f6c34d327172472c5b419d7}\\
			\hline
			&Input scripts&There is no input script since the coins are newly generated.\\
			\hline
			&Output scripts&There are three output scripts: one specifying the receiver of the mined bitcoin, one using OP\_RETURN to push data onto the blockchain (not decodable as text), and one that seems to have produced an error when decoding the script (but similarly to the second one, it transfers no coins).\\
			\hline
			&Inferences from structure&This is a transaction claiming the mining reward of (at the time) $\sim12.6$BTC (including transaction fees). This can be inferred as there is no input script, thus this transaction is the miner's claiming their reward.\\
			\hline
			2&blockchain.com link&\href{https://www.blockchain.com/btc/tx/4a3751b29128bea458f7f07d4f92b29322ba1900bbc364130bf2648e73f8cfad}{https://www.blockchain.com/btc/tx/4a3751b29128bea458f7f07d4f92b2932\newline 2ba1900bbc364130bf2648e73f8cfad}\\
			\hline
			&Input scripts&There is a single input script, which includes a witness.\\
			\hline
			&Output scripts&There are a large number of output scripts, most simply checking the hash of the receiver's address is valid, and a few additionally verifying and checking the signature.\\
			\hline
			&Inferences from structure&As this transaction has a single input, multiple outputs, and pays a large sum ($\sim1.7$ BTC), I suspect that this is payout from a mining guild, or possibly a sale of bitcoin by an exchange.\\
			\hline
			3&blockchain.com link&\href{https://www.blockchain.com/btc/tx/496cda9b0f4083b7775822f07a1541176866947b9f93006752de029489e1abd6}{https://www.blockchain.com/btc/tx/496cda9b0f4083b7775822f07a1541176\newline 866947b9f93006752de029489e1abd6}\\
			\hline
			 &Input scripts&This transaction contains a large number of input scripts. Most of these have a small input amount, with three being notably large: $\sim1, \sim4.3,$ and $\sim25$ bitcoin. All input scripts are from the same address.\\
			\hline
			 &Output scripts&The transaction contains two output scripts. One claims $\sim13.2$ BTC, and the other $22.5$ BTC.\\
			\hline
			 &Inferences from structure&Due to the amounts involved, and the exact payment of $22.5$ BTC to the second output address, I suspect that this transaction is a user selling BTC to an exchange. This is supported by investigating the recipient address of the $22.5$ BTC, which shows it being involved in numerous high-value transactions, where the address seems to mainly be used as a middle man, often transferring the exact input amount onwards to another address.\\
			\hline
		\end{tabu}
		\caption{Bitcoin-Testnet Transactions Details}
		\label{table_bitcoin_testnet_transactions}
	\end{table}
	
	\section{Investment Advice: Bitcoin, NXT or Gold}
	
	The judgment of what to invest in depends on your reasons for investing. Firstly, if you are investing to maintain value then gold would be the best candidate. The value of gold rarely fluctuates, and it has some value as a material outside of being used as currency. It is also the most durable of these options, as it is not reliant on any technology or network in order to have worth and be useable. Even in extreme circumstances such as financial collapse, gold will maintain value as a currency, while any cryptocurrency may see its value plummet or become invalid due to collapse of its network or userbase.
	
	If you are investing to make short-term returns, then NXT is likely the best candidate. Over the past few months both Bitcoin and NXT have started to increase in value. Out of these two NXT has risen faster, and due to its technical advantages over Bitcoin, such as using Proof of Stake rather than Proof of Work making it less wasteful to operate and more expensive to attack, and its ability to trade any asset on its network, it has more utility than Bitcoin. This theoretically gives NXT an edge over Bitcoin, meaning it should steal market share and users from it over time. However, this is not necessarily going to happen due to the popularity of Bitcoin. Even though Bitcoin is an inefficient and outdated cryptocurrency, the fact that it was the first coin to gain widespread popularity and support cannot be ignored. It currently holds the largest market cap of any cryptocurrency, and is unlikely to lose popularity quickly.
	
	Additionally, there is a large number of cryptocurrencies available, most of which have at least some technical merits over Bitcoin, meaning NXT is competing with these as well as Bitcoin in the attempt to gain market share and value. This means it is is extremely unlikely for NXT to become a true Bitcoin equal, and its value is likely to stay volatile. This means that investing in NXT could see short-term profits, but these are unlikely to be large or sustainable.
	
	The popularity of Bitcoin may also be a detriment when it comes to investment, however. Due to its technical shortcomings, Bitcoin is only capable of processing a handful of transactions per second, with long delays before a transaction is confirmed. This means that the more popular the cryptocurrency becomes, the more expensive transactions will become due to fees needed to ensure the transaction is confirmed within a reasonable time period. This will likely cap the potential market cap and value of Bitcoin in the future, which limits the investment opportunities in it. Even worse, this effect may result in people becoming aware of its technical shortcomings, and moving to other cryptocurrencies, leading to a massive crash in users, market cap, and value.
	
	Because of the reasons outlined above, I would recommend investing in NXT, as it has the largest potential to yield returns on the investment in the long-term.
	
	\pagebreak
	
	\section{An analysis of XRP (Ripple)}
	
	\subsection{Idea and Justification for Creation}
	
	XRP, commonly referred to as Ripple, is a decentralized cryptocurrency used as part of the XRP Ledger. It was initially released in 2012, intended as a currency exchange and transfer system for financial institutions \cite{finder}. Ripple Labs Inc., the developer of XRP, uses it as part of their solutions and offerings to financial institutions. The idea behind its creation was to allow fast, cheap, and reliable international transactions, which currently can take up to 5 business days and charge as much as 6\% commission fees \cite{topcryptocurrencies}. Any currency beyond XRP can be represented and traded on the XRP Ledger as `issued currencies.` These can be traded for each other, or for XRP, and can represent any item, not just traditional currency.
	
	\subsection{Technical Differences to Bitcoin}
	
	XRP differs significantly from Bitcoin. Firstly, unlike in Bitcoin and many other blockchain technologies, nodes do not need to store the entire history of the blockchain in order to track its current state. XRP uses a ledger system, where each ledger stores the entire current state of the system \cite{rippledevelopers}. This means that the cost in terms of storage for each node is reduced. Every ledger is produced by a set of transactions that are applied to the previous ledger. The new ledger is then summarized using a hash tree and compared across the nodes \cite{bitcoinmagazine}. A consensus is reached using the XRP Ledger Consensus Protocol (XRP LCP), which ensures that all nodes agree on the set of transactions, followed by a transaction processing protocol that ensures that all nodes produce the same ledger based on this set \cite{Chase}. The XRP LCP is designed to ensure that all users of the XRP Ledger reach an agreement on the state of the ledger, all valid transactions are processed, transactions can be processed even if some participants are missing or behaving nefariously, and to avoid requiring large amounts of resources to be spent \cite{rippledevelopers}. This is a large difference from Bitcoin, which requires large amounts of processing power (and thus energy) to be spent for each new block to be produced.
	
	Additionally, anyone can set up a system to mine Bitcoin and be a part of creating the next block. However, XRP uses validated servers to reach the consensus to produce the next ledger. Only trusted servers, those present in a participant's Unique Node List (UNL) are considered when determining if a given transaction or ledger is valid. While participants can freely choose which servers are in their UNL, Ripple Labs published a recommended UNL. This presents a barrier to entry for any party wanting to join the consensus system, and makes XRP much less decentralized than Bitcoin and other cryptocurrencies.
	
	Another major difference between Bitcoin and XRP is in the transaction fees. In Bitcoin, transaction fees are voluntarily added by the source of the transaction, and any fees included in transactions in a mined block are collected by the miner in full. In practice, transaction fees are mandatory if a user wants their transaction included within a reasonable amount of time. The typical transaction fee is completely controlled by market forces and varies with the network congestion. For reference, the estimated fee for a transaction to be included in the next block is $\sim\$0.30$ \cite{bitcoinfees}. On the other hand, XRP has a mandated minimum transaction fee of $0.00001$ XRP per transaction built into the protocol (at the time of writing this is equivalent to roughly $\$0.0000031$). Additionally, this fee is burned, rather than being collected as in Bitcoin. The minimum transaction fee scales with the current load on the network. In XRP, this transaction fee serves two purposes: firstly, it discourages attacks on the network attempting to overload it with transactions, both intentional and unintentional \cite{}. Secondly, in burning the XRP paid for the fee it makes all remaining XRP more valuable. This makes XRP a deflationary currency: not only is there a finite supply of the coin, the supply is constantly (slowly) decreasing.
	
	In addition to the differences above, XRP is also much faster and has a higher throughput than Bitcoin. The XRP Ledger is one of the fastest blockchains currently, being able to confirm a transaction in under little as four seconds \cite{investinblockchain}. Bitcoin, on the other hand, can take longer than an hour before a transaction is considered confirmed, as a transaction is usually only considered to be completed when it has been present in six consecutive blocks. XRP can currently handle up to $1500$ transactions per second, giving it a much higher transaction capacity than Bitcoin, which can only process $\sim7$ transactions per second \cite{blocksplain}. This makes XRP much better suited as a transaction processing system and currency than Bitcoin. Additionally, according to Ripple Labs XRP has the capability to scale to handle as many transactions per second as Visa ($50000$) \cite{ripple}.
	
	\subsection{Mining Ripple}
	
	Unlike Bitcoin, Ripple is not designed to be mined, and no reward is given to nodes for helping produce the next ledger \cite{cointelegraph}. Instead, 100 billion XRP were initially created by its creator, Ripple Labs. Out of these, roughly 38 billion are currently available with the rest being held by Ripple Labs, most of them in escrow to be released at a rate of 1 billion per month \cite{finder}. This also means that there is no financial incentive to operating a validator server as discussed above, as there is no reward in XRP for doing so. Instead, the reward given to validators is in being a part in controlling the network, and deciding on the future evolution of the network. Ripple Labs' justification for this being a viable system is that the only cost to running a server is the electricity it consumes, which is extremely low compared to a Bitcoin mining rig - roughly equivalent to running an email server \cite{technicalFAQ}. In addition to this, institutions or individuals that rely on the XRP Ledger, for example for financial transactions, are incentivised to operate servers in order to ensure its reliability and stability. If already runnning a server, the additional cost to run a validator is negligible \cite{technicalFAQ}. However, this design also further reduces the extent to which XRP is decentralized.
	
	\subsection{Performance}
	
	Despite its technical advantages over Bitcoin, XRP has failed to outperform the more popular coin. It briefly held a price of $\sim\$0.2$ for the second half of 2017, before rapidly climbing to a peak value of $\$3.84$ in early 2018, and rapidly falling down to just below $\$1$. Since this peak, the price of XRP in USD has been on a slow but steady decline. Bitcoin has similarly been declining after a massive peak, but with more fluctuations in value and market cap, and has seen a small increase in the last month, which XRP has failed to mimic. XRP currently holds a fairly stable market cap of $\sim\$13$ billion, while Bitcoin is at nearly $\$80$ billion \cite{coinmarketcap}.
	
	While XRP is not performing as well as Bitcoin recently, compared to the general cryptocurrency market it is performing favourably. It remains at the number 3 spot in terms of market cap, being only $\$1$ billion behind Ethereum at number two, and comfortably $\sim\$9.5$ billion ahead of Litecoin at number four. Additionally, most cryptocurrencies have seen a small slump over the past few days, which XRP has begun to recover from faster than many others.
	
	\subsection{Notable Attacks and Events}
	
	Ripple has so far managed to stay out of the limelight and has not been the victim of any major attacks. According to Ripple Labs' CTO David Schwartz, this is because of the consensus protocol used in the XRP Ledger, which makes it immune to `51\%' attacks that many other cryptocurrencies have suffered from \cite{newsbtc}. However, this does not mean that XRP is without controversy. One potential worry is the fact that Ripple Labs owns $\sim60$ billion of the $100$ billion XRP in existence. Most of these ($\sim55$ billion) are locked in escrow accounts, scheduled to be released $1$ billion per month over the next $55$ months. Ripple Labs have stated that the reason for their holding this many XRP is to incentivize the company to continue development of XRP Ledger to the best of their ability \cite{technicalFAQ}, but many users worry that this gives them extreme powers over the cryptocurrency, breaking the principles of decentralization. For example, Ripple Labs have the ability to drastically affect the price of XRP by releasing a large amount of their holdings, at any time \cite{investinblockchain}. Another source of worry is the design choices limiting decentralization discussed previously, such as most validated servers being controlled by financial institutions. This goes against a common drive behind interest in cryptocurrencies and blockchain, namely independence from banks and governments.
	
	A major event that affected XRP occurred in May of 2018, when Ripple Labs was named in a lawsuit filed in California, USA. This lawsuit alleged that Ripple Labs ran a scheme to collect hundreds of millions of dollars by generating XRP and trading them in unregistered sales in a ``never-ending initial coin offering'' \cite{bloomberg}. The lawsuit is based on the idea that XRP act more like stock for Ripple Labs than a traditional cryptocurrency, and as such should be subject to more strict controls and regulations \cite{blockexplorer}. As of November 2018, the lawsuit is still ongoing.
	
	\subsection{Assessment of XRP and Conclusion}
	
	While XRP has many technological advantages over Bitcoin, such as drastically faster transaction confirmation speeds and higher transaction throughput, many design choices that contribute to these feats may severely restrict its potential as a cryptocurrency. Although XRP technically is a decentralized cryptocurrency, in practice this ideal is violated by the fact that Ripple Labs control $\sim62\%$ of all XRP, and ledger consensus is carried out by a select few parties, and there are large barriers to becoming one of these parties. Lastly, while many see XRP's connection to Ripple Labs and their financial services as a positive regarding its feasibility and stability, XRP is not necessary to the operation of the XRP Ledger due to the issued currencies mentioned earlier. Thus, this connection has no impact on the value of XRP as a currency.
	
	\bibliography{bibliography}
	
\end{document}